
% \iffalse
\let\negmedspace\undefined
\let\negthickspace\undefined
\documentclass[journal,12pt,twocolumn]{IEEEtran}
\usepackage{cite}
\usepackage{amsmath,amssymb,amsfonts,amsthm}
\usepackage{algorithmic}
\usepackage{graphicx}
\usepackage{textcomp}
\usepackage{xcolor}
\usepackage{txfonts}
\usepackage{listings}
\usepackage{enumitem}
\usepackage{mathtools}
\usepackage{gensymb}
\usepackage{comment}
\usepackage[breaklinks=true]{hyperref}
\usepackage{tkz-euclide} 
\usepackage{listings}
\usepackage{gvv}
%def\inputGnumericTable{}                        
\usepackage[latin1]{inputenc}                              
\usepackage{color}                                            
\usepackage{array}                                            
\usepackage{longtable}                                       
\usepackage{calc}                                             
\usepackage{multirow}                                         
\usepackage{hhline}                                           
\usepackage{ifthen}                                           

\usepackage{lscape}
\usepackage{tabularx}
\usepackage{array}
\usepackage{float}
\usepackage{multicol}

\newtheorem{theorem}{Theorem}[section]
\newtheorem{problem}{Problem}
\newtheorem{proposition}{Proposition}[section]
\newtheorem{lemma}{Lemma}[section]
\newtheorem{corollary}[theorem]{Corollary}
\newtheorem{example}{Example}[section]
\newtheorem{definition}[problem]{Definition}
\newcommand{\BEQA}{\begin{eqnarray}}
\newcommand{\EEQA}{\end{eqnarray}}
\newcommand{\define}{\stackrel{\triangle}{=}}
\theoremstyle{remark}
\newtheorem{rem}{Remark}

% Marks the beginning of the document
\begin{document}
\bibliographystyle{IEEEtran}
\vspace{3cm}

\title{5.A.JEE ADVANCED/IIT-JEE1.Fill in the Blanks}
\author{ai24btech11020 K.Rishika}

\maketitle
\bigskip   
\renewcommand{\thefigure}{\theenumi}
\renewcommand{\thetable}{\theenumi}     
\begin{enumerate}[start=1]
\item The larger of $99^{50}+100^{50}$ and $101^{50}$ is......
	\hfill\textbf{(1982- 2 Marks)}\\
\item The sum of the coefficients of the polynomial $(1+x-3x^2)^2163$ is .......
	\hfill\textbf{(1982- 2 Marks)}\\
\item If $(1+ax)^n=1+8x+24x^2+...$ then a=.... and n=.......
	\hfill\textbf{(1983- 2 Marks)}\\
\item Let n be positive integer. If the coefficients of 2nd, 3rd, and 4th terms in the expansion of $(1 + x)^n)$ are in A.P, then the value of n is.......
	\hfill\textbf{(1994- 2 Marks)}\\
\item The sum of the rational terms in the expansion of $(\sqrt{2}+3^\frac{1}{5})^{10}$ is.......
	\hfill\textbf{(1997- 2 Marks)}
\end{enumerate}
\end{document}
