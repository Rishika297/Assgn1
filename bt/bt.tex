% \iffalse
\let\negmedspace\undefined
\let\negthickspace\undefined
\documentclass[journal,12pt,twocolumn]{IEEEtran}
\usepackage{cite}
\usepackage{amsmath,amssymb,amsfonts,amsthm}
\usepackage{algorithmic}
\usepackage{graphicx}
\usepackage{textcomp}
\usepackage{xcolor}
\usepackage{txfonts}
\usepackage{listings}
\usepackage{enumitem}
\usepackage{mathtools}
\usepackage{gensymb}
\usepackage{comment}
\usepackage[breaklinks=true]{hyperref}
\usepackage{tkz-euclide} 
\usepackage{listings}
                                  
\def\inputGnumericTable{}                                 
\usepackage[latin1]{inputenc}                                
\usepackage{color}                                            
\usepackage{array}                                            
\usepackage{longtable}                                       
\usepackage{calc}                                             
\usepackage{multirow}                                         
\usepackage{hhline}                                           
\usepackage{ifthen}                                           
\usepackage{lscape}

\newtheorem{theorem}{Theorem}[section]
\newtheorem{problem}{Problem}
\newtheorem{proposition}{Proposition}[section]
\newtheorem{lemma}{Lemma}[section]
\newtheorem{corollary}[theorem]{Corollary}
\newtheorem{example}{Example}[section]
\newtheorem{definition}[problem]{Definition}
\newcommand{\define}{\stackrel{\triangle}{=}}
\theoremstyle{remark}
\newtheorem{rem}{Remark}
\begin{document}

\bibliographystyle{IEEEtran}
\vspace{3cm}

\title{5.A.JEE ADVANCED/IIT-JEE1.Fill in the Blanks}
\author{ai24btech11020 K.Rishika}
\begin{enumerate}
\maketitle
\item[1.] The larger of $99^{50}+100^{50}$ and $101^{50}$ is......\\[2pt]
	\hfill{(1982- 2 Marks)}\\[6pt]
\item[2.]The sum of the coefficients of the polynomial $(1+x-3x^2)^2163$ is .......\\[2pt]
	\hfill{(1982- 2 Marks)}\\[6pt]
\item[3.]If $(1+ax)^n=1+8x+24x^2+...$ then a=.... and n=.....\\[2pt]
	\hfill{(1983- 2 Marks)}\\[6pt]
\item[4.]Let n be positive integer. If the coefficients of 2nd, 3rd, and 4th terms in the expansion of $(1 + x)^n)$ are in A.P, then the value of n is.....\\[2pt]
	\hfill{(1994- 2 Marks)}\\[6pt]:
\item[5.]The sum of the rational terms in the expansion of $(\sqrt{2}+3^\frac{1}{5})^{10}$ is......\\[2pt]
	\hfill{(1997- 2 Marks)}
\bigskip
\renewcommand{\thefigure}{\theenumi}
\renewcommand{\thetable}{\theenumi}
\end{enumerate}
\end{document}
