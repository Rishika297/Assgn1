
%iffalse
\let\negmedspace\undefined
\let\negthickspace\undefined
\documentclass[journal,12pt,twocolumn]{IEEEtran}
\usepackage{cite}
\usepackage{amsmath,amssymb,amsfonts,amsthm}
\usepackage{algorithmic}
\usepackage{graphicx}
\usepackage{textcomp}
\usepackage{xcolor}
\usepackage{txfonts}
\usepackage{listings}
\usepackage{enumitem}
\usepackage{mathtools}
\usepackage{gensymb}
\usepackage{comment}
\usepackage[breaklinks=true]{hyperref}
\usepackage{tkz-euclide} 
\usepackage{listings}
\usepackage{gvv}      
\parindent 0px
%\def\inputGnumericTable{}                                 
\usepackage[latin1]{inputenc}                                
\usepackage{color}                                            
\usepackage{array}                                            
\usepackage{longtable}                                       
\usepackage{calc}                                             
\usepackage{multirow}                                         
\usepackage{hhline}                                           
\usepackage{ifthen}                                           
\usepackage{lscape}
\usepackage{tabularx}
\usepackage{array}
\usepackage{float}


\newtheorem{theorem}{Theorem}[section]
\newtheorem{problem}{Problem}
\newtheorem{proposition}{Proposition}[section]
\newtheorem{lemma}{Lemma}[section]
\newtheorem{corollary}[theorem]{Corollary}
\newtheorem{example}{Example}[section]
\newtheorem{definition}[problem]{Definition}
\newcommand{\BEQA}{\begin{eqnarray}}
\newcommand{\EEQA}{\end{eqnarray}}
\newcommand{\define}{\stackrel{\triangle}{=}}
\theoremstyle{remark}
\newtheorem{rem}{Remark}

% Marks the beginning of the document
\begin{document}
\bibliographystyle{IEEEtran}
\vspace{3cm}

\title{3.Quadratic Equation and Inequations(Inequalities)}\textbf{}
\author{ai24btech11020 - Rishika Kotha}
\begin{enumerate}
\maketitle
\item[24.] The equation $e^{\sin x}$-$e^{-\sin x}$-4=0 has:\hspace{1cm}\textcolor{red}{[2012]}\\[6pt]
        (a)   infinite number of real roots\\[2pt]
        (b)   no real roots\\[2pt]
        (c)   exactly one real root\\[2pt]
        (d)   exactly four roots\\[6pt]
\item[25.] The real number k for which the equation, $2x^3+3x+4=0$ has two distinct real roots in [0,1]\hspace{4.6cm}\textcolor{red}{[JEE M 2013]}\\[6pt]
        (a)  lies between 1 and 2\\[2pt]
        (b)  lies between 2 and 3\\[2pt]
        (c)  lies between -1 and 0\\[2pt]
        (d)  does not exist.\\[6pt]
\item[26.] The number of values of k, for which the system of equations:\hspace{2cm}\textcolor{red}{[JEE M 2013]}\\[6pt]
     (k+1)x+8y=4k\\[2pt]
     kx+(k+3)y=3k-1\\[2pt]
has no solution, is\\[2pt]
(a)  infinite\hspace{2.4cm}(b)  1\\[2pt]
(c)  2\hspace{3.4cm}(d)  3\\[6pt]
\item[27.] If the equations $x^2+2x+3=0$ and $ax^2+bx+c=0$,a,b,c$\in$ R , have a   common   root , then a:b:c is\hspace{2.3cm}\textcolor{red}{[JEE M 2013]}\\[6pt]
(a)  1:2:3\hspace{2.7cm}(b)  3:2:1\\[2pt]
(c)  1:3:2\hspace{2.7cm}(d)  3:1:2\\[6pt] 
\item[28.]If a $\in$ R and the equation  $-3(x-[x])^2 +2(x-[x])+a^2 =0$ (where [x] denotes the greatest integer $\leq$ x) has no integral solution, then all possible values of a lie in the interval:\\[2pt]
\phantom{2cm}\hspace{4.8cm}\textcolor{red}{[JEE M 2014]}\\[6pt]
(a)  (-2,-1)\hspace{2.5cm}(b)  $(-\infty,2)\cup(2,\infty)$\\[2pt]
(c)  $(-1,0)\cup(0,1)$\hspace{1.1cm}(d)  (1,2)\\[6pt]
\item[29.]Let\quad$\alpha$ and $\beta$ be the roots of equation $px^2 + qx +r = 0$, $p\neq0$. If p,q,r are in A.P. and $\frac{1}{\alpha}+\frac{1}{\beta}=4$, then the value of $|\alpha-\beta|$ is:[6pt] \phantom{2cm}\hspace{4.8cm}\textcolor{red}{[JEE M 2014]}\\[6pt]
(a)  $\frac{\sqrt{34}}{9}$\hspace{5cm}(b)  $\frac{2\sqrt{13}}{9}$\\[2pt]
(c)  $\frac{\sqrt{61}}{9}$\hspace{5cm}(d)$\frac{2\sqrt17}{9}$\\[6pt]
\item[30.] Let $\alpha$ and $\beta$ be the roots of the equation $x^2-6x-2=0$. If $a_n=\alpha^n-\beta^n$, for $n\geq1$, then the value of $\frac{a_{10}-2a_8}{2a_9}$ is equal to:\\[6pt]
\phantom{2cm}\hspace{4.8cm}\textcolor{red}{[JEE M 2015]}\\[6pt]
(a)  3\hspace{5cm}(b)-3\\[2pt]
(c)  6\hspace{5cm}(d)-6\\[6pt]
\item[31.] The sum of all real values of x satisfying the equation $(x^2-5x+5)^{x^2+4x+60}=1$ is :\\[6pt]\phantom{2cm}\hspace{4.8cm}\textcolor{red}{[JEE M 2016]}\\[6pt]
(a)  6\hspace{5cm}(b)5\\[2pt]
(c)  3\hspace{5cm}(d)-4\\[6pt]
\item[32.]If $\alpha,\beta\in C$ are the distinct roots, of the equation $x^2-x+1=0$ , then $\alpha^{101}+\beta^{107}$ is equal to :\hspace{3.8cm}\textcolor{red}{[JEE M 2018]}\\[6pt]
(a)  0\hspace{5cm}(b)  1\\[2pt]
(c)  2\hspace{5cm}(d)  -1\\[6pt]
\item[33.]Let p,q$\in$ R. If $2-\sqrt{3}$ is a root of the quadratic equation, $x^2+px+q=0$,then:\\[6pt]
\phantom{2cm}\hspace{2.8cm}\textcolor{red}{[JEE M 2019- 9 April(M)]}\\[6pt]
(a)  $p^2-4q+12=0$ \hspace{1cm}(b)$q^2-4p-16=0$\\[2pt]
(c)  $q^2+4p+14=0$ \hspace{1cm}(d)$p^2-4q-12=0$\\[6pt]

\bigskip

\renewcommand{\thefigure}{\theenumi}
\renewcommand{\thetable}{\theenumi}

\end{enumerate}
\end{document}

